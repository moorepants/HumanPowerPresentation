\documentclass[]{beamer}

%\usepackage{pgfpages}
\usepackage{beamerthemesplit}
\usepackage{graphicx}
\usepackage{multimedia}
\usepackage{hyperref}

%\setbeameroption{show notes on second screen}
%\setbeamertemplate{blocks}[rounded][shadow=true]
\xdefinecolor{lgray}{cmyk}{0,0,0,0.05}
\setbeamercolor{uppercol}{fg=white,bg=structure}
\setbeamercolor{lowercol}{fg=black,bg=lgray}
\setbeamertemplate{navigation symbols}{}
\xdefinecolor{tuBlue}{rgb}{0,0.65,0.84}
\usecolortheme[named=tuBlue]{structure}

%\logo{\includegraphics[height=1cm]{tu-delft.jpg}}

\title{Use of Human Power in the Developing World}
\author[Jason~K.~Moore]{Reverend~Jason~K.~Moore}
\institute[UC Davis]{Department of Mechanical and Aerospace Engineering\\University of California, Davis}
%\date{October 9, 2009}
\begin{document}

\frame{\titlepage}
\frame{\tableofcontents}

\section{Introduction}
\subsection{History}
\frame
{
    \frametitle{Historical Uses of Human Power}
    \begin{columns}[t]
        \column{5cm}
        \begin{overprint}
        Transportation:
        \begin{itemize}
            \item<1->\alert<1>{walking and running}
            \item<1->\alert<2>{hand carts}
            \item<1->\alert<3>{rowing boats}
            \item<1->\alert<4>{bicycle}
        \end{itemize}
        Tools:
        \begin{itemize}
            \item<1->\alert<5>{plow}
            \item<1->\alert<6>{water pump}
            \item<1->\alert<7>{food processors}
            \item<1->\alert<8>{lathes, saws, sewing}
            \item<1->\alert<9>{spinning}
        \end{itemize}
        \end{overprint}
        \column{6cm}
		\begin{overprint}
			\begin{figure}[]
  				\centering
  				\includegraphics[height=4cm]<1>{longWalk.jpg}
  				\includegraphics[height=4cm]<2>{rickshaw.jpg}
  				\includegraphics[height=4cm]<3>{galley.jpg}
  				\includegraphics[height=4cm]<4>{bicycle.jpg}
  				\includegraphics[height=4cm]<5>{plow.jpg}
  				\includegraphics[height=4cm]<6>{waterPump.jpg}
  				\includegraphics[height=6cm]<7>{mill.png}
  				\includegraphics[height=6cm]<8>{lathe.png}
  				\includegraphics[height=6cm]<9>{SpinningWheel.jpg}
  			\end{figure}
		\end{overprint}
        \end{columns}
}
\note{From the dawn of humankind we have relied on human power to get things
done. Our bodies are amazing machines alone and we can perform feats of
strength and agility that surpass many mechanical machines. As soon as tools
were invented we began the search for the optimal use of our muscle power.}
\subsection{Human Power}
\frame
{
    \frametitle{Power and Energy}
    \begin{itemize}
        \item<1-> We are energy transformers!
        \item<2-> We digest our fuel (food) to supply our brain and muscles
            with the energy need to think and move.
    \end{itemize}
    \onslide<3->
	\begin{beamerboxesrounded}[upper=uppercol,lower=lowercol,shadow=true]{Work:
        a measure of energy}
		\begin{center}
            $Work = force \cdot distance$ $(Joules=Newtons\cdot Meters)$ 
		\end{center}
	\end{beamerboxesrounded}
    \begin{itemize}
        \item<4-> Power is the measure of how fast we can do work
    \end{itemize}
    \onslide<5->
	\begin{beamerboxesrounded}[upper=uppercol,lower=lowercol,shadow=true]{Power}
		\begin{center}
            $Power=\frac{Work}{Time}$ $(Watt = \frac{Joules}{Seconds})$
        \end{center}
	\end{beamerboxesrounded}
}
\note{Human bodies are built to transform energy. We covert food stuff to both
electrical and mechanical (maybe spiritual too) energy so that we can think and
compute and so that we can use our muscles to do work. There are two important Formally, work is the
measure of energy and is a force times a distance. If I apply a force to this
table and it moves a certain distance, the product of those numbers is a
measure of how much energy is used to move the table. }
\frame
{
    \begin{center}
        \includegraphics[width=7cm]<1>{powerVSduration.png}
        \includegraphics[width=7cm]<2>{oneHour.png}
        \includegraphics[width=7cm]<3>{fiveHours.png}
    \end{center}
}
\frame
{
    \frametitle{Efficiency}
    \begin{itemize}
        \item<1-> We always lose energy when converting from one type to another.
        \item<2-> Efficiency is the measure of this loss.
    \end{itemize}
    \onslide<3->
	\begin{beamerboxesrounded}[upper=uppercol,lower=lowercol,shadow=true]{Efficiency}
        \begin{center}
            $input\cdot efficiency=output$
        \end{center}
    \end{beamerboxesrounded}
    \onslide<4->
    \begin{itemize}
        \item Most efficient means of mechanical work: pedaling or rowing
    \end{itemize}
    \begin{columns}[t]
        \column{5.5cm}
        \includegraphics[width=2cm]{pedaling.jpg}
        \column{5.5cm}
        \includegraphics[width=3cm]{rowing.jpg}
    \end{columns}
}
\frame
{
    \frametitle{How efficient are we?}
    \begin{itemize}[<+->]
        \item ICE max efficiency is \structure{35\%}, but more like \alert{18\% to 20\%}
        \item Electric motors/generators are \alert{65\% to 95\%} efficient
        \item Humans are around \alert{18\% to 26\%} efficient in turning food energy to work
        \item Efficiencies stack by multiplication
    \end{itemize}
}
\frame
{
    \frametitle{Pedaling Rates}
    \begin{itemize}
        \item Optimum pedal and rowing rates for efficient power output
        \item Usually need gear reduction or levers to optimize the power
    \end{itemize}
}
\section{Machine Design}
\subsection{Transforming Human Motion}
\frame
{
    \frametitle{Pedaling to Rotational}
    \begin{columns}[t]
        \column{6cm}
        \begin{itemize}
            \item<1->\alert<1>{Chain drives: 90\%+}
            \item<1->\alert<2>{Shaft drives: 80-90\%}
            \item<1->\alert<3>{Flat belt drives: 90\%+}
            \item<1->\alert<4>{Friction drives: $<$80\%}
        \end{itemize}
        \column{5cm}
		\begin{overprint}
			\begin{figure}[]
  				\centering
  				\includegraphics[height=4cm]<1>{chain.jpg}
  				\includegraphics[height=4cm]<2>{bevel.jpg}
  				\includegraphics[height=4cm]<3>{flatbelt.jpg}
  				\includegraphics[height=4cm]<4>{friction.jpg}
  			\end{figure}
		\end{overprint}
    \end{columns}
}
\frame
{
    \frametitle{Rotational to Electrical}
    \begin{columns}[t]
        \column{6cm}
        \begin{itemize}[<+->]
            \item Rotational generators are most common: 65-95\%
            \item Difficult to find low speed generators
            \item DC generators: voltage is proportional to the speed
            \item Alternators: minimum excitation needed, but easy to find
        \end{itemize}
        \column{5cm}
		\begin{overprint}
			\begin{figure}[]
 				\centering
  				\includegraphics[height=4cm]<1-3>{generator.jpg}
  				\includegraphics[height=4cm]<4->{alternator.jpg}
  			\end{figure}
		\end{overprint}
    \end{columns}
}
\subsection{Energy Storage}
\frame
{
    \frametitle{Energy Storage Types}
    \begin{itemize}[<+->]
        \item Springs store potential energy \structure{$E=\frac{1}{2}kx^2$}
        \item Flywheels store kinetic energy \structure{$E=\frac{1}{2}I\omega^2$}
        \item Capacitors store energy like a spring \structure{$E=\frac{1}{2}CV^2$}
        \item Batteries create energy from a chemical reaction and store energy
        \item They all act as an energy buffer
    \end{itemize}
}
\section{Real World Applications}
\subsection{Successful Projects}
\frame
{
    \frametitle{World Bicycle Relief}
}
\note{}
\frame
{
    \frametitle{Water Pumps}
    \begin{center}
        \includegraphics[width=7cm]{women_at_hand_pump_well.jpg}
    \end{center}
}
\note{Hand operated water pumps (hand pump) are some of the most widely used, reliable and
appropriate human powered machines.
\url{http://en.wikipedia.org/wiki/Hand_pump}}
\frame
{
    \frametitle{Kickstart Water Pumps}
    \begin{center}
        \begin{figure}[]
            \includegraphics[width=7cm]{kickstart.jpg}
        \end{figure}
        \url{www.kickstart.org}
    \end{center}
}
\note{KickStart’s Original MoneyMaker pump was introduced in September 1996.
This small treadle operated pump could pull water from as deep as 23 feet (7m)
and be used to furrow irrigate up to two acres of land. It was superseeded buy
the Super-Money Maker in 1999 to provide hose and sprinkely based watering that
could be pumped uphill. Primarily used in East Africa.}
\frame
{
    \frametitle{The Full Belly Project}
    \begin{center}
        \begin{figure}[]
            \includegraphics[width=7cm]{fullbelly.jpg}
        \end{figure}
        \url{www.thefullbellyproject.org}
    \end{center}
}
\frame
{
    \frametitle{R2B2 by Christoph Thetard}
    \begin{center}
        \begin{figure}[]
            \includegraphics[width=3.5cm]{R2B2_01.png}
            \includegraphics[width=3.5cm]{R2B2_02.png}
        \end{figure}
        \url{www.christoph-thetard.de}
    \end{center}
}
\subsection{My Projects}
\frame
{
    \frametitle{ZAmbulance and wheelchairs in Zambia, Africa}
    \begin{columns}[t]
        \column{6cm}
        \begin{itemize}[<+->]
            \item Short distance transport for patients
            \item Materials are imported and very expensive
            \item Only NGO's can purchase and distribute
        \end{itemize}
        \column{5cm}
		\begin{overprint}
			\begin{figure}[]
  				\centering
  				\includegraphics[height=4cm]<1>{zambulance.jpg}
  				\includegraphics[height=4cm]<2>{wheelchair.jpg}
  				\includegraphics[height=4cm]<3>{handTrike.jpg}
  			\end{figure}
		\end{overprint}
    \end{columns}
}
\frame
{
    \frametitle{Human powered machines in Guatemala}
    \begin{columns}[t]
        \column{6cm}
        \begin{itemize}[<+->]
            \item Corn grinding for masa
            \item Rope water pump
            \item Macadamia nut husker
            \item Clothes washing machine
            \item Peanut sheller
        \end{itemize}
        \column{5cm}
		\begin{overprint}
		\begin{figure}[]
  			\centering
  				\includegraphics[height=4cm]<1>{cornGrinder.jpg}
  				\includegraphics[height=4cm]<2>{waterPumpMaya.jpg}
  				\includegraphics[height=4cm]<3>{macadamiaHusker.jpg}
  				\includegraphics[height=4cm]<4>{wash.jpg}
  				\includegraphics[height=4cm]<5>{peanutSheller.jpg}
  			\end{figure}
		\end{overprint}
    \end{columns}
}
\frame
{
    \frametitle{UC Davis Human Powered Utility Vehicle}
    \begin{center}
        \begin{figure}[]
            \includegraphics[width=7cm]{mike.jpg}
        \end{figure}
    \end{center}
}
\frame
{
    \frametitle{Pedal Desk}
    \begin{columns}[t]
        \column{6cm}
        \begin{itemize}
            \item Power a laptop with pedal power
            \item Educate students on power usage
            \item \href{http://news.bbc.co.uk/2/hi/science/nature/8394055.stm}{What does it take to power a home?}
        \end{itemize}
        \column{5cm}
        \begin{center}
        \begin{figure}[]
            \includegraphics[width=4cm]{ped_desk.jpg}
        \end{figure}
        \end{center}
    \end{columns}
}
\subsection{Example case}
\frame
{
    \frametitle{Knife Sharpener}
    \begin{columns}[t]
        \column{6cm}
        \begin{itemize}[<+->]
            \item Grinders are used as a common method for sharpening knives
            \item Electric grinder or mechanical?
        \end{itemize}
        \column{5cm}
		\begin{overprint}
			\begin{figure}[]
  				\centering
  				\includegraphics[width=4cm]<1>{ugandaGrinder02.jpg}
  				\includegraphics[width=4cm]<2>{indiaGrinder.jpg}
  			\end{figure}
		\end{overprint}
    \end{columns}
}
\frame
{
    \frametitle{Thanks for your time!}
    \begin{center}
        \begin{figure}[]
            \includegraphics[width=5cm]{lineTrack.jpg}
        \end{figure}
    \end{center}
}
\end{document}
