\documentclass[]{beamer}

%\usepackage{pgfpages}
\usepackage{beamerthemesplit}
\usepackage{graphicx}
\usepackage{multimedia}
\usepackage{hyperref}

% uncomment these for various versions of the notes
%\setbeameroption{show notes on second screen}
%\setbeameroption{show notes}
%\setbeameroption{show only notes}
%\setbeamertemplate{blocks}[rounded][shadow=true]
\xdefinecolor{lgray}{cmyk}{0,0,0,0.05}
\setbeamercolor{uppercol}{fg=white,bg=structure}
\setbeamercolor{lowercol}{fg=black,bg=lgray}
\setbeamertemplate{navigation symbols}{}
\xdefinecolor{tuBlue}{rgb}{0,0.65,0.84}
\usecolortheme[named=tuBlue]{structure}

%\logo{\includegraphics[height=1cm]{tu-delft.jpg}}

\title{Use of Human Power in the Developing World}
\author[Jason~K.~Moore]{Rev.~Dr.~Jason~K.~Moore}
\institute[UC Davis]{Department of Mechanical and Aerospace Engineering\\University of California, Davis}
%\date{October 9, 2009}
\begin{document}

\frame{\titlepage}
\frame{\tableofcontents}

\section{Introduction}
\subsection{History}
\frame
{
    \frametitle{Historical Uses of Human Power}
    \begin{columns}[t]
        \column{5cm}
        \begin{overprint}
        Transportation:
        \begin{itemize}
            \item<1->\alert<1>{walking and running}
            \item<1->\alert<2>{hand carts}
            \item<1->\alert<3>{rowing boats}
            \item<1->\alert<4>{bicycle}
        \end{itemize}
        Tools:
        \begin{itemize}
            \item<1->\alert<5>{plow}
            \item<1->\alert<6>{water pump}
            \item<1->\alert<7>{food processors}
            \item<1->\alert<8>{lathes, saws, sewing}
            \item<1->\alert<9>{spinning}
        \end{itemize}
        \end{overprint}
        \column{6cm}
		\begin{overprint}
			\begin{figure}[]
  				\centering
  				\includegraphics[height=4cm]<1>{images/longWalk.jpg}
  				\includegraphics[height=4cm]<2>{images/rickshaw.jpg}
  				\includegraphics[height=4cm]<3>{images/galley.jpg}
  				\includegraphics[height=4cm]<4>{images/bicycle.jpg}
  				\includegraphics[height=4cm]<5>{images/hand-plow-women-1919.jpg}
  				\includegraphics[height=4cm]<6>{images/waterPump.jpg}
  				\includegraphics[height=6cm]<7>{images/mill.png}
  				\includegraphics[height=6cm]<8>{images/lathe.png}
  				\includegraphics[height=6cm]<9>{images/SpinningWheel.jpg}
  			\end{figure}
		\end{overprint}
        \end{columns}
}
\note{From the dawn of humankind we have relied on human power to get things
done. Our bodies are amazing machines and for most of our history, the best
machines we had available to perform manual tasks. Our bodies have always been
used for transportation, they essentially evolved to move us about in a very
efficient manner. As, tools and machines began to arrive on the scene began the
search for the optimal use of our muscle power. }

\subsection{Human Power}
%
\frame
{
    \frametitle{The Human Machine}
    \begin{itemize}
        \item<1-> We are energy transformers!
        \item<2-> We digest our fuel (food) to supply our brain and muscles
            with the energy need to think and move.
        \item<3-> About as efficient as an Internal Combustion Engine (ICE) (35\%)
    \end{itemize}
    \onslide<3->
    \begin{center}
        \includegraphics[width=3cm]{images/HumanMuscles.png}
        \Large{$=$}
        \includegraphics[width=3cm]{images/internal-combustion-engine.jpg}
    \end{center}
}
\note{In a way, we are simply another type of energy covnerter. We convert
energy from food to primarily mechanical energy. Some believe that the
evolution of us as a complex machine was simply to optimize our motion and
movement capabilities. But just as any energy conversion process, energy is
lost along the way. Human's are not especially fundamentally energy efficient,
so we have to think carafully aobut  how we maximize the efficiency of the work
we do?}
%
\frame
{
    \frametitle{Energy and Power}
    \begin{itemize}
        \item<1-> Work is a measure of energy
    \end{itemize}
    \onslide<1->
	\begin{beamerboxesrounded}[upper=uppercol,lower=lowercol,shadow=true]{Work:
        a measure of energy}
		\begin{center}
            $Work = force \cdot distance$ $(Joules=Newtons\cdot Meters)$ 
		\end{center}
	\end{beamerboxesrounded}
    \begin{itemize}
        \item<2-> Power is the measure of how fast we can do work
    \end{itemize}
    \onslide<3->
	\begin{beamerboxesrounded}[upper=uppercol,lower=lowercol,shadow=true]{Power}
		\begin{center}
            $Power=\frac{Work}{Time}$ $(Watt = \frac{Joules}{Seconds})$
        \end{center}
	\end{beamerboxesrounded}
}
\note{Human bodies are built to transform energy. We covert food stuff to both
electrical and mechanical (maybe spiritual too) energy so that we can think and
compute and so that we can use our muscles to do work. There are two important
concepts that I want to reenforce from Physics 101 that aren't always
understood (I blame it partially on the energy companies confusing use of
kilowatt-hours). Work is a measure of energy and can simply be described as the
force times the distance it acts over. If I apply a force to this table and it
moves a certain distance, the product of those numbers is a measure of how much
energy is used to move the table. Power is how fast we do work, or work per
unit of time. If I take all day to push this desk across the room versus taking
30 secs, then the former takes much less power.}
%
\frame
{
    \begin{center}
        \includegraphics[width=7cm]<1>{images/powerVSduration.png}
        \includegraphics[width=7cm]<2>{images/oneHour.png}
        \includegraphics[width=7cm]<3>{images/fiveHours.png}
				\\
				Wilson2004
    \end{center}
}
\note{\small{So now that we are refreshed about work and power, I'd like to show you
one of my favorite graphs about human power output. This graph is generated
from series of experiments in which the person generates a fixed amount of
power until they are completely exhausted. On the X axis is time to exhaustion
on a non-linear scale. The Y axis is the power generated during the test. The
lowest curve is for a nominal healthy man and the curves and data points above
that are for various elite atheletes. I'm not sure were an average person from
a malnurished place may fall, but certainly below the healthy man line. Notice
the slope of the curves for durations under and hour and how steep they are.
To drive this graph home some let's see how long a heatly man can generate 200
watts of power (e.g. to power two 100 watt lightbulbs). He can last about 1
hour, until he is completely exhausted. But check out if we lower the power
output to 100 watts, the duration increases to about 5 hours. 100 watts is
about how much we expend when bicycling at an average commuting pace and is a
good expected maximum power output for many tasks. Keep in mind that these
experiments attempt to measure the raw power output of the humans, but that raw
power output has to be converted into some useful mechanical motion. Can anyone
guess how many people generating 100 watts would it take to power UC Davis's
electrical consumption? Something like 270,000 people.}}
%
\frame
{
    \frametitle{Efficiency}
    \begin{itemize}
        \item<1-> Energy is lost when converting from one type to another.
        \item<2-> Efficiency is a measure of this loss.
    \end{itemize}
    \onslide<3->
	\begin{beamerboxesrounded}[upper=uppercol,lower=lowercol,shadow=true]{Efficiency}
        \begin{center}
					$efficiency= \frac{output}{input}$
        \end{center}
    \end{beamerboxesrounded}
    \onslide<4->
    \begin{itemize}
        \item Pedaling and rowing: most efficient at moderate to high power.
    \end{itemize}
    \begin{columns}[t]
        \column{5.5cm}
        \includegraphics[width=2cm]{images/pedaling.jpg}
        \column{5.5cm}
        \includegraphics[width=3cm]{images/rowing.jpg}
    \end{columns}
}
\note{We all know that in the real world, there is always energy lost in
processes and we are on a constant mission to reduce these energy losses. This
is especially important when designing power extracting machines for humans,
due to the limited power they can generate. The efficiency of a process is the
output divided by the input. We almost always need the efficiency to be as high
as possible. It turns out that for human power generation, rowing and pedaling
are the most efficient methods of generating moderate to maximum power. They
both make use of our largest muscles, the legs and cyclic motions.}
%
\frame
{
    \frametitle{How efficient are we?}
		\begin{tabular}{ll}
			Thing & Efficiency\\
			\hline
			Human (food to mechanical) & \alert{18\% to 26\%}\\
			\hline
			IC Engine & theorectial maximum: \structure{35\%},\\
			& reality: \alert{18\% to 20\%}\\
			\hline
			Electric motors & \alert{65\% to 95\%}\\
			\hline
			Transmissions & \alert{75\% to 99\%}
		\end{tabular}
		\vskip1cm
    \onslide<2->
		\begin{beamerboxesrounded}[upper=uppercol,lower=lowercol,shadow=true]{Efficiency}
			\begin{center}
				Efficiencies stack by multiplication!
			\end{center}
		\end{beamerboxesrounded}
}
\note{This table gives some idea of how efficient we are compared to other
systems. Human's are only 18 to 26 percent efficient in converting their food
to mechanical energy. An adult's nominal metabolic rate of energy consumption
is 100-150 watts. We're producing 3kW right here in the room, doing nothing.
The IC Engine is very similar in efficiency to a human and electrical motors
are pretty efficient. Keep in mind that efficiencies mutliply when you stack
devices. A human pedaling a generator through a transmission has an efficiency
from food to electricity of something like 15\%.}
%
\frame
{
	\frametitle{Pedaling Rates}
	\begin{center}
		\begin{columns}[t]
			\column{5cm}
			\includegraphics[width=5cm]{images/power-vs-cadence-red.png}
			\\
			Wilson2004
			\column{5cm}
			\includegraphics[width=5cm]{images/motor-power-vs-rpm-red.png}
		\end{columns}
		\begin{center}
			\includegraphics[width=4.5cm]{images/motor-efficiency-vs-rpm-red.png}
		\end{center}
	\end{center}
}
\note{So how does this efficiency and power generation play out in the design
of human power harvesting machines? Take a look at these curves. The first is a
typical power versus pedaling rate for a cyclist. Notice that the maximum power
is produced somewhere around 85 rpm. Imagine that you want to have the person
generate electricity with a DC generator. The plot on the right shows the power
curve for a typical motor. The maximum power output here is around 1500 rpm.
And on top of that the most efficient rpm for the motor is around 2300 rpm. A
machine that couples the rider to the generator will have to be designed to
take all of this and more into account. First graph taken from Wilson2004 and
motor graphs from:
http://www.recumbents.com/mars/pages/proj/sadler/assist/projsadassist.html}
%
\section{Machine Design}
\subsection{Transforming Human Motion}
\frame
{
    \frametitle{Pedaling to Rotational}
    \begin{columns}[t]
        \column{6cm}
        \begin{itemize}
            \item<1->\alert<1>{Chain drives: 90\%+}
            \item<1->\alert<2>{Shaft drives: 80-90\%}
            \item<1->\alert<3>{Flat belt drives: 90\%+}
            \item<1->\alert<4>{Friction drives: $<$80\%}
        \end{itemize}
        \column{5cm}
		\begin{overprint}
			\begin{figure}[]
  				\centering
  				\includegraphics[height=4cm]<1>{images/chain.jpg}
  				\includegraphics[height=4cm]<2>{images/bevel.jpg}
  				\includegraphics[height=4cm]<3>{images/flatbelt.jpg}
  				\includegraphics[height=4cm]<4>{images/friction.jpg}
  			\end{figure}
		\end{overprint}
    \end{columns}
}
\frame
{
    \frametitle{Rotational to Electrical}
    \begin{columns}[t]
        \column{6cm}
        \begin{itemize}[<+->]
            \item Rotational generators are most common: 65-95\%
            \item Difficult to find low speed generators
            \item DC generators: voltage is proportional to the speed
            \item Alternators: minimum excitation needed, but easy to find
        \end{itemize}
        \column{5cm}
		\begin{overprint}
			\begin{figure}[]
 				\centering
  				\includegraphics[height=4cm]<1-3>{images/generator.jpg}
  				\includegraphics[height=4cm]<4->{images/alternator.jpg}
  			\end{figure}
		\end{overprint}
    \end{columns}
}
\subsection{Energy Storage}
\frame
{
    \frametitle{Energy Storage Types}
    \begin{itemize}[<+->]
        \item Springs store potential energy \structure{$E=\frac{1}{2}kx^2$}
        \item Flywheels store kinetic energy \structure{$E=\frac{1}{2}I\omega^2$}
        \item Capacitors store energy like a spring \structure{$E=\frac{1}{2}CV^2$}
        \item Batteries create energy from a chemical reaction and store energy
        \item They all act as an energy buffer
    \end{itemize}
}
\section{Real World Applications}
\subsection{Successful Projects}
%
\frame
{
    \frametitle{The Bicycle}
    \begin{center}
        \includegraphics[width=4cm]<1>{images/BeijingBicycles.jpg}
        \includegraphics[width=7cm]<2>{images/tires.png}
    \onslide<2->
		\\
    \url{www.alaindelorme.com}
    \end{center}
}
\note{If there is one human powered machine that just can't be matched, it is
the bicycle. The bicycle is the most energy efficient means of transportation.
It is even more efficient than walking! Bicycles are the most abundant
transportation machine on the planet and they are utilitize in a multiude of
ways to move people and goods. The bicycle is a very valuable item in most of
the developing world. Many people think that an inexpensive and durable bicycle
can be a life saver for people in developing nations, allowing for better
commerce, kids getting to school, etc. There are an array of projects that are
trying to better lives with bicycles. But the bicycle is still a distant
commodity for many people as the price is not within reach. If anyone can
create a less than \$50 durable bicycle for the developing world, they will get
the Nobel prize.}
%
\frame
{
    \frametitle{Water Pumps}
    \begin{center}
        \includegraphics[width=7cm]{images/women_at_hand_pump_well.jpg}
    \end{center}
}
\note{Hand operated water pumps (hand pump) are some of the most widely used, reliable and
appropriate human powered machines. The classic water pump's design is
practically indestructable and parts are available in the most remote
locations allowing well drawn water to be brought to communities around the
world. Your designs should aspire to be as simple.
\url{http://en.wikipedia.org/wiki/Hand_pump}}
%
\frame
{
    \frametitle{Kickstart Water Pumps}
    \begin{center}
        \begin{figure}[]
            \includegraphics[width=7cm]{images/kickstart.jpg}
        \end{figure}
        \url{www.kickstart.org}
    \end{center}
}
\note{KickStart’s Original MoneyMaker pump was introduced in September 1996.
This small treadle operated pump could pull water from as deep as 23 feet (7m)
and be used to furrow irrigate up to two acres of land. It was superseeded buy
the Super-Money Maker in 1999 to provide hose and sprinkely based watering that
could be pumped uphill. Primarily used in East Africa.}
%
\frame
{
    \frametitle{The Full Belly Project}
    \begin{center}
        \begin{figure}[]
            \includegraphics[width=7cm]{images/fullbelly.jpg}
        \end{figure}
        \url{www.thefullbellyproject.org}
    \end{center}
}
\note{The Full belly project introduced the universal nut sheller in 2005,
orginally designed to shell peanuts. It can shell 50 kg of peanuts per hour
compared to 1.5 kg by hand in an hour. It has a simple design using only
concrete and some metal parts and should last 20 years with little maintenance.
The simplicity of this design is the major reason it has been so succesful.}
%
\frame
{
	\frametitle{One Laptop Per Child}
	\begin{center}
		\begin{columns}[t]
			\column{5cm}
			\includegraphics[width=5cm]{images/OLPC.jpg}
			\column{5cm}
			\includegraphics[width=5cm]{images/olpc-tablet-crank.jpg}
		\end{columns}
		\begin{center}
			\includegraphics[width=4.5cm]{images/one-laptop-per-child-tablet-solar.jpg}
		\end{center}
	\end{center}
}
\note{The OLPC project launched around 2006 with the goal of introducing an
inexpensive computer to children in the developing world. They've worked on
leveraging the rapid decrease in power consumption in electronics to develop an
extremely low power laptop (on the order of 2 watts or less). They've developed
a hand crank to charge/run it and are working on integrated solar. Side note:
research is being done where they leave computers in rural places and find that
there is rapid self learning.}
%
\frame
{
	\frametitle{Low power electronics}
	\begin{center}
		\begin{columns}[t]
			\column{5cm}
			\includegraphics[width=5cm]{images/shaker-flashlight.jpg}
			\column{5cm}
			\includegraphics[width=5cm]{images/super-battery-hand-crank-USB-charger.jpg}
		\end{columns}
	\end{center}
}
\note{There are a number of super low power devices that can be power by the
user. The shaker type flashlight is very popular, along with an assortment of
radios and other lights. Arjen Jansen has recently published his dissertation
on the subject of these devices. And his country is the home of the Watt dance
floor where the dancers produce the club's lighting.}
%
\frame
{
	\frametitle{Rock The Bike}
	\begin{center}
		\begin{columns}[t]
			\column{5cm}
			\includegraphics[width=5cm]{images/rock-the-bike-pedal-stage.jpg}
			\column{5cm}
			\includegraphics[width=5cm]{images/rock-the-bike-ice-cream-bike-02.jpg}
		\end{columns}
	\end{center}
}
\note{Rock the Bike is a bay area baed group that started up developing ways to
power sound systems with human power. They have music concerts that are pedal
powered by the audience and have even sent touring bands out only be bicycle
and setup to power all there own sound equipment. They also have a sweet
bicycle blender and now even an ice cream maker.}
%
\frame
{
    \frametitle{Green Gyms}
    \begin{center}
				\includegraphics[width=5cm]{images/green-revolution.jpg}
    \end{center}
}
%
\frame
{
    \frametitle{R2B2 by Christoph Thetard}
    \begin{center}
        \begin{figure}[]
            \includegraphics[width=3.5cm]{images/R2B2_01.png}
            \includegraphics[width=3.5cm]{images/R2B2_02.png}
        \end{figure}
        \url{www.christoph-thetard.de}
    \end{center}
}
\note{This universal kitchen appliance is a pretty elegant design. Many people
have worked on various forms of human powered machines with interchangeable
devices.}
%
\frame
{
    \frametitle{iRock Rocking Chair}
    \begin{center}
        \begin{figure}[]
            \includegraphics[width=6cm]{images/irock_rocking_chair.jpg}
        \end{figure}
        \url{http://www.treehugger.com/gadgets/irock-rocking-chair-charges-your-apple-device.html}
    \end{center}
}
\note{Clever way to extract energy from rocking.}
%
\frame
{
    \frametitle{Piezoelectric Dance Floor}
    \begin{center}
        \begin{figure}[]
            \includegraphics[width=6cm]{images/piezo-dance-floor.jpg}
        \end{figure}
    \end{center}
}
\note{Piezoelectric floor tiles generate enough energy from dancing to power
LEDs in the floor.}
%
\frame
{
    \frametitle{Electricity generating backpack}
    \begin{center}
        \begin{figure}[]
            \includegraphics[width=6cm]{images/electric-backpack.jpg}
        \end{figure}
        \url{http://www.lightningpacks.com}
    \end{center}
}
\note{Recovers electricity from normal walking.}
%
\subsection{My Projects}
\frame
{
    \frametitle{ZAmbulance and wheelchairs in Zambia, Africa}
    \begin{columns}[t]
        \column{6cm}
        \begin{itemize}[<+->]
            \item Short distance transport for patients
            \item Materials are imported and very expensive
            \item Only NGO's can purchase and distribute
        \end{itemize}
        \column{5cm}
		\begin{overprint}
			\begin{figure}[]
  				\centering
  				\includegraphics[height=4cm]<1>{images/zambulance.jpg}
  				\includegraphics[height=4cm]<2>{images/wheelchair.jpg}
  				\includegraphics[height=4cm]<3>{images/handTrike.jpg}
  			\end{figure}
		\end{overprint}
    \end{columns}
}
\frame
{
    \frametitle{Human powered machines in Guatemala}
    \begin{columns}[t]
        \column{6cm}
        \begin{itemize}[<+->]
            \item Corn grinding for masa
            \item Rope water pump
            \item Macadamia nut husker
            \item Clothes washing machine
            \item Peanut sheller
        \end{itemize}
        \column{5cm}
		\begin{overprint}
		\begin{figure}[]
  			\centering
  				\includegraphics[height=4cm]<1>{images/cornGrinder.jpg}
  				\includegraphics[height=4cm]<2>{images/waterPumpMaya.jpg}
  				\includegraphics[height=4cm]<3>{images/macadamiaHusker.jpg}
  				\includegraphics[height=4cm]<4>{images/wash.jpg}
  				\includegraphics[height=4cm]<5>{images/peanutSheller.jpg}
  			\end{figure}
		\end{overprint}
    \end{columns}
}
\frame
{
    \frametitle{UC Davis Human Powered Utility Vehicle}
    \begin{center}
        \begin{figure}[]
            \includegraphics[width=7cm]{images/mike.jpg}
        \end{figure}
    \end{center}
}
\frame
{
    \frametitle{Mobile Ministry Unit}
    \begin{center}
			\includegraphics[width=7cm]{images/mmu.jpg}
    \end{center}
}
\frame
{
    \frametitle{Pedal Desk}
    \begin{columns}[t]
        \column{6cm}
        \begin{itemize}
            \item Power a laptop with pedal power
            \item Educate students on power usage
						\item \url{http://mae.ucdavis.edu/~biosport/jkm/ped_desk.htm}
        \end{itemize}
        \column{5cm}
        \begin{center}
        \begin{figure}[]
            \includegraphics[width=4cm]{images/ped_desk.jpg}
        \end{figure}
        \end{center}
    \end{columns}
}
\note{We built this desk to demonstrate how much power it actually takes to run
a laptop. Turns out that it would take 250,000 people to power UCD's electric
needs.}
\frame
{
    \frametitle{How many people does it take to power a home?}
    \url{http://www.youtube.com/watch?v=C93cL_zDVIM}
}
\subsection{Example case}
\frame
{
    \frametitle{Whipped Cream}
		\begin{center}
			\includegraphics[width=4cm]{images/strawberry-whipped-cream.jpg}
			\\
			\includegraphics[width=3cm]{images/hand-beater.jpg}
			\large{VERSUS}
			\includegraphics[width=3cm]{images/electric-beater.jpg}
		\end{center}
}
\begin{frame}[shrink=30]
	\begin{center}
		\alert{Reverend Jason Moore (moorepants@gmail.com)}\\
	\end{center}
	Resources:
	\nocite{McCullagh1977}
	\nocite{Wilson1986}
	\nocite{Wilson2004}
	\nocite{Dean2008}
	\nocite{Jansen2011}

	\bibliographystyle{unsrt}
	\small\bibliography{bicycle}
\end{frame}
\end{document}
